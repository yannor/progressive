%%=============================================================================
%% Conclusie
%%=============================================================================

\chapter{Conclusie}
\label{ch:conclusie}

%% TODO: Trek een duidelijke conclusie, in de vorm van een antwoord op de
%% onderzoeksvra(a)g(en). Wat was jouw bijdrage aan het onderzoeksdomein en
%% hoe biedt dit meerwaarde aan het vakgebied/doelgroep? Reflecteer kritisch
%% over het resultaat. Had je deze uitkomst verwacht? Zijn er zaken die nog
%% niet duidelijk zijn? Heeft het onderzoek geleid tot nieuwe vragen die
%% uitnodigen tot verder onderzoek?

Het web zal altijd winnen. Net zoals het Flash versloeg en apps op de desktop zal het ook native apps verslaan.

Programmeurs willen niet meerdere talen leren zodat ze een app kunnen maken voor elke store apart. Bedrijven willen geen extra geld betalen om extra mensen aan te nemen voor ondersteuning op alle platformen.
Iedereen wilt één plaats voor hun code. Eénmaal schrijven, één plaats om te onderhouden. Als oplossing is er hiervoor progressive web apps.

Progressive web apps hebben in mijn ogen vele voordelen tegenover native apps. Momenteel is dit nog maar het begin maar het zal enkel nog groeien omdat ze kunnen bieden wat native apps niet kunnen. 

Native apps hebben momenteel nog steeds de overhand, maar dit zal veranderen. Uiteindelijk gaan steeds meer en meer bedrijven overschakelen naar progressive web apps. Momenteel is het al mogelijk je progressive web app in de store te zetten. Hierdoor kunnen bedrijven ook hier zichtbaar worden. 

Voor ik aan deze thesis begon had ik wel gedacht dat in de toekomst progressive web apps invloed zouden kunnen hebben op de toekomst van native apps, nu, na deze thesis, denk ik dat dit er sneller zal zijn als verwacht. Door de vele voordelen die een progressive web app biedt zal dit een veel gekozen middel zijn om zich te tonen op de markt. Uiteindelijk zal het native apps van de troon stoten. Zeker als het steun blijft krijgen van grote spelers zoals Google. 


Bij mijn onderzoeksvragen heb ik me de vraag gesteld wat de voordelen zijn van een progressive web app tegenover een native app. Ook heb ik gekeken naar de ontwikkeltijd tussen beide. 

Initieel is het makkelijker om een native app te maken. Dit kan dan wel maar draaien op één platform. Hierdoor is het makkelijker om initieel een wat grotere inspanning te leveren voor een progressive web app waardoor je geen verschillende programmeertalen moet leren om je app op de  verschillende stores te krijgen.
Ik verwachtte voor een progressive web app een grotere leercurve. Ondanks de weinige ervaring die ik had met web apps was het iets dat snel te leren was aangezien er geen specifieke taal voor is. 

Qua voordelen kan progressive web apps meer bieden dan een native app, maar toch wegen de voordelen van native app momenteel zwaarder door. Native apps zijn gekend bij het publiek. Ze willen iets, ze gaan naar de store. Progressive web apps zijn nog niet gekend genoeg bij het publiek. Mettertijd zal dit veranderen. Naargelang de progressive web app meer en meer gebruikt zal worden zullen meer mensen de voordelen gaan inzien. 


