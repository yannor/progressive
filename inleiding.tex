%%=============================================================================
%% Inleiding
%%=============================================================================

\chapter{Inleiding}
\label{ch:inleiding}

“The full Safari engine is inside of iPhone. And so, you can write amazing Web 2.0 and Ajax apps that look exactly and behave exactly like apps on the iPhone. And these apps can integrate perfectly with iPhone services. They can make a call, they can send an email, they can look up a location on Google Maps. And guess what there is no SKD that you need. You got everything you need if you know how to write apps using the most modern web standards to write amazing apps for the IPhone today” ~\autocite{keynote2007}

Steve Jobs schetste in zijn speech in 2007 al een idee omtrent wat we tegenwoordig progressive web apps ofwel PWA's noemen. Hierbij stelde hij Apple's internetbrowser Safari voor waarop hij een idee gaf wat er allemaal mogelijk mee is. In 2008 introduceerde Apple de App Store waardoor het idee rond progressive web apps meer op de achtergrond raakte.

De voorbije jaren zijn er 3 grote spelers op de mobiele markt geweest, Apple, Windows en Android. Elk van deze hebben hun eigen store waar ze apps aanbieden en waar je als ontwikkelaar je apps op kunt lanceren. Elke store werkt met een eigen programmeertaal.
\begin{itemize}  
	\item Android: Java
	\item Apple: Recent overgeschakeld van Objective-C naar Swift
	\item Windows: C\#
\end{itemize}

Als ontwikkelaar wil je geen 3 verschillende programmeertalen leren. Als bedrijf wil je niet 3 verschillende mensen huren voor hetzelfde werk te doen.

Als antwoord hierop heeft Google op de "Google I/O, the developer conference" PWA's voorgesteld. Progressive web apps. Google beschrijft de progressive web app als "applicaties die gebruik maken van nieuwe technologiën om zo het beste van native apps en mobiele website naar de gebruiker te brengen. Ze zijn betrouwbaar en snel".
Met andere woorden, Google wilt webapplicaties die de voordelen van native apps gaan gebruiken zodat de webapp zich gaat gedragen als een native app. 
Als ontwikkelaar zou je dan eenmaal je webapp moeten maken en kan deze op eender welk platform bekeken en gebruikt worden. Je zou geen verschillende programmeertalen moeten leren om dit voor elk platform apart te maken en toch zal het op elk platform lijken alsof het een native app is, speciaal gemaakt voor Android, Apple, ...


Uit een studie van \textcite{comScore} in 2016 is gebleken dat gebruikers 87\% van hun tijd op de smarthphone of tablet doorbrengen op een mobiele app terwijl ze maar 13\% van hun tijd doorbrengen op het web. Als je deze cijfers ziet zou je je kunnen afvragen of het wel slim is je tijd te investeren in een progressive web app. Is het niet beter je tijd te spenderen aan het ontwikkelen van native apps? Je hebt miscchien wat meer werk om het op elk platform apart te kunnen krijgen maar als je cijfers ziet van de gebruikers lijkt het erop dat het wel zijn vruchten kan afwerpen.
(zie figuur \ref{fig:usage}).



\section{Stand van zaken}
\label{sec:stand-van-zaken}

%% TODO: deze sectie (die je kan opsplitsen in verschillende secties) bevat je
%% literatuurstudie. Vergeet niet telkens je bronnen te vermelden!

\lipsum[7-20]

\section{Probleemstelling en Onderzoeksvragen}
\label{sec:onderzoeksvragen}

%% TODO:
%% Uit je probleemstelling moet duidelijk zijn dat je onderzoek een meerwaarde
%% heeft voor een concrete doelgroep (bv. een bedrijf).
%%
%% Wees zo concreet mogelijk bij het formuleren van je
%% onderzoeksvra(a)g(en). Een onderzoeksvraag is trouwens iets waar nog
%% niemand op dit moment een antwoord heeft (voor zover je kan nagaan).

Als bedrijf wil je gemakkelijke gevonden worden door iedereen. Wat voor bedrijf je ook hebt, of je nu spelletjes maakt online, of je nu een bakkerij hebt of een groentenwinkel, een groot bedrijf met honderden werknemers of een eenmanszaak. Als potentiële klanten iets zoeken wat jij aanbiedt, dan wil je dat ze u zo snel en makkelijk mogelijk kunnen vinden. Maar hoe kan je dit het best doen? 

Hulpvragen:
\begin{itemize}  
	\item ... todo
	\item Wat zijn de verschillen van PWA tegenover native app?
\end{itemize}



\section{Opzet van deze bachelorproef}
\label{sec:opzet-bachelorproef}

%% TODO: Het is gebruikelijk aan het einde van de inleiding een overzicht te
%% geven van de opbouw van de rest van de tekst. Deze sectie bevat al een aanzet
%% die je kan aanvullen/aanpassen in functie van je eigen tekst.

De rest van deze bachelorproef is als volgt opgebouwd:

In Hoofdstuk~\ref{ch:methodologie} wordt de methodologie toegelicht en worden de gebruikte onderzoekstechnieken besproken om een antwoord te kunnen formuleren op de onderzoeksvragen.

%% TODO: Vul hier aan voor je eigen hoofstukken, één of twee zinnen per hoofdstuk

In Hoofdstuk~\ref{ch:conclusie}, tenslotte, wordt de conclusie gegeven en een antwoord geformuleerd op de onderzoeksvragen. Daarbij wordt ook een aanzet gegeven voor toekomstig onderzoek binnen dit domein.

